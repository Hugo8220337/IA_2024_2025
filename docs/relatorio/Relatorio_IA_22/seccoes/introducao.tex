\section{Introdução}
\subsection{Contextualização}
Este trabalho foi realizado para o âmbito da disciplina de Inteligência Artificial na Escola Superior de Tecnologia e Gestão do Instituto Politécnico do Porto, e tem como objetivo utilizar os conhecimentos adquiridos no decorrer das aulas teóricas e práticas para a realização do treinamento de vários modelos para um bot que irá jogar o jogo Pokemon HeartGold.

\subsection{Objetivos do Trabalho}
Este trabalho tem como principal objetivo o desenvolvimento de uma aplicação que, através de modelos de visão computacional baseados na arquitetura YOLO (You Only Look Once), e mais especificamente modelos de deteção de objetos, seja capaz de automatizar alguma tarefa num jogo, sendo o jogo escolhido o Pokemon HeartGold.

Este trabalho tem um foco especial na área de Machine Learning e visão por computador, sendo trabalhado as seguintes competências:

\begin{itemize}
    \item A modelação do conhecimento existente no domínio de um problema e no seu espaço de solução, com vista à sua utilização computacional;
    \item A geração, otimização e avaliação de soluções válidas para um problema complexo, utilizando uma perspetiva iterativa de desenvolvimento e validação;
    \item A análise e comparação crítica de diferentes abordagens, com vista à seleção da mais adequada à resolução do problema;
    \item A melhoria iterativa e incremental de uma abordagem para a resolução de um problema com base em resultados passados;
    \item A criação de datasets para problemas específicos de Machine Learning;
    \item A utilização de algoritmos de Machine Learning para treino de modelos, e a utilização em produção
\end{itemize}

\subsection{Estrutura do Relatório}
Este relatório está organizado em várias secções que refletem as etapas realizadas no desenvolvimento do projeto. Na Secção \ref{definicao_problema}, é feita a definição do problema, onde se explica o processo de escolha do jogo, as tarefas a automatizar e as restrições impostas ao projeto. A Secção \ref{recolha_imagens} descreve a recolha de imagens necessárias para o treino dos modelos, detalhando as estratégias adotadas e os problemas enfrentados. A Secção \ref{etiquetagem} aborda o processo de etiquetagem das imagens, com indicação da ferramenta utilizada e as classes definidas.

A Secção \ref{modelos_treinados} apresenta os modelos de visão computacional treinados, incluindo as versões utilizadas e os resultados preliminares. Seguidamente, a Secção \ref{data_augmentation} descreve as técnicas de data augmentation aplicadas para melhorar o desempenho dos modelos. A Secção \ref{avaliacoes_dos_modelos} trata da avaliação dos modelos, apresentando as métricas utilizadas, os resultados obtidos e a comparação entre diferentes versões, culminando na escolha do modelo final.

Na Secção \ref{implementacao_producao}, é descrita a implementação da aplicação em ambiente de produção, com destaque para as funcionalidades desenvolvidas e as limitações encontradas. Por fim, a Secção \ref{conclusao} apresenta a conclusão do trabalho, com uma reflexão sobre as aprendizagens obtidas, dificuldades sentidas e possíveis melhorias futuras.